\documentclass{beamer}

\usepackage{amssymb,amsmath}
\usepackage{relsize}
\usepackage{hyperref}
\usepackage{minted}
\usepackage{color}
\usepackage{xpatch,letltxmacro}
\LetLtxMacro{\cminted}{\minted}
\let\endcminted\endminted
\xpretocmd{\cminted}{\RecustomVerbatimEnvironment{Verbatim}{BVerbatim}{}}{}{}
\renewcommand{\figurename}{Listing}

\title{Using crypto in Haskell}
\author{Sharif~Olorin \url{<sio@tesser.org>}}
\institute{Ambiata}

\begin{document}

\frame{\titlepage}

\begin{frame}[fragile]

\centering
\begin{cminted}[mathescape]{c}
#include <stddisclaimer.h>
\end{cminted}

\end{frame}

\begin{frame}[fragile]

\begin{minted}[mathescape]{sh}
# cloc usr/src/openssl | head                                                                                                                  
                                                                                                                                            -----------------------------------------------------------------------------------
Language     files      blank    comment     code
-----------------------------------------------------------------------------------
C              867      33658      33632    249878

# cloc usr/src/hs-tls | head                                                                                                                  
                                                                                                                                            -----------------------------------------------------------------------------------
Language     files      blank    comment     code
-----------------------------------------------------------------------------------
Haskell        69        1393      1199      8518
\end{minted}


\end{frame}

\begin{frame}

\center
...so why so many terrible libraries?
\end{frame}

\begin{frame}
\frametitle{A basic authentication framework}

\begin{itemize}
\item Securely store user credentials.
\item Implement authentication without leaking data.
\end{itemize}

\end{frame}

\begin{frame}

\begin{figure}
  \includegraphics[scale=0.55]{img/auth.png}
\end{figure}
\end{frame}


\begin{frame}[fragile]

\begin{minted}[mathescape]{haskell}
newtype User = User Text

newtype Password = Password Text

newtype Salt = Salt ByteString

newtype Hash = Hash ByteString

data Credential = Credential !Salt !Hash

data Verification = Verified | NotVerified
\end{minted}

\end{frame}


\begin{frame}[fragile]

\begin{minted}[mathescape]{haskell}
authenticate :: User -> Password -> IO Verification

register :: User -> Password -> IO ()
\end{minted}

\end{frame}

\begin{frame}[fragile]

\begin{minted}[mathescape]{haskell}
register :: User -> Password -> IO ()
register username pass = do
  salt <- readEntropy
  storeUser salt $ hash salt pass
\end{minted}

\end{frame}

\begin{frame}[fragile]

\begin{minted}[mathescape]{haskell}
authenticate :: User -> Password -> IO Verification
authenticate username pass =
  lookupUser username >>= \(mu :: Maybe Credential) ->
    case mu of
      Just cred ->
        pure $ verify pass cred
\end{minted}

\end{frame}

\begin{frame}[fragile]

\begin{minted}[mathescape]{haskell}
authenticate :: User -> Password -> IO Verification
authenticate username pass =
  lookupUser username >>= \(mu :: Maybe Credential) ->
    case mu of
      Just cred ->
        pure $ verify pass cred
      Nothing ->
        pure $ verify "" fakeCred
\end{minted}

\end{frame}

\begin{frame}[fragile]

\begin{minted}[mathescape]{haskell}
authenticate :: User -> Password -> IO Verification
authenticate username pass =
  lookupUser username >>= \(mu :: Maybe Credential) ->
    case mu of
      Just cred ->
        verify pass cred
      Nothing -> do
        _ <- verify pass fakeCred
        pure NotVerified
\end{minted}

\end{frame}

\begin{frame}[fragile]

\begin{minted}[mathescape]{haskell}
verify :: Password -> Credential -> IO Verification
verify pass (Credential salt h) = do
  h' <- hash salt pass
  case h `constEq` h' of
    True ->
      pure Verified
    False ->
      pure NotVerified
\end{minted}

\end{frame}


\begin{frame}

\begin{figure}
  \includegraphics[scale=0.55]{img/auth-prims.png}
\end{figure}
\end{frame}



\begin{frame}[fragile]

\frametitle{On comparison}

\begin{minted}[mathescape]{haskell}
-- | Message authentication code wrapper.
newtype MAC = MAC ByteString
  deriving (Eq, Show)

-- | Verify origin and integrity of message.
authenticate :: MAC -> Message -> Key -> Verification
authenticate mac msg key =
  let
    mac' = computeMAC msg key
  in
  case mac == mac' of
    True ->
      Verified
    False ->
      NotVerified
\end{minted}

\end{frame}

\begin{frame}[fragile]

\begin{minted}[mathescape]{haskell}
eq :: ByteString -> ByteString -> Bool
eq a@(PS fp off len) b@(PS fp' off' len')
  | len /= len'              = False
  | fp == fp' && off == off' = True
  | otherwise                = compareBytes a b == EQ

compareBytes :: ByteString -> ByteString -> Ordering
compareBytes (PS fp1 off1 len1) (PS fp2 off2 len2) =
    accursedUnutterablePerformIO $
      withForeignPtr fp1 $ \p1 ->
      withForeignPtr fp2 $ \p2 -> do
        i <- memcmp (p1 `plusPtr` off1)
                    (p2 `plusPtr` off2)
                    (min len1 len2)
        return $! case i `compare` 0 of
                    EQ  -> len1 `compare` len2
                    x   -> x
\end{minted}

\end{frame}

\begin{frame}[fragile]

\begin{minted}[mathescape]{c}
bool const_cmp(uint8_t *buf1,
               size_t s1,
               uint8_t *buf2,
               size_t s2) {
	size_t i;
	uint8_t acc = 0;
	if (s1 != s2) {
		return FALSE;
	}
	for (i = 0; i < s1; i++) {
		acc |= buf1[i] ^ buf2[i];
	}
	if (acc == 0) {
		return TRUE;
	}
	return FALSE;
}
\end{minted}

\end{frame}

\begin{frame}

\frametitle{Testing}

\begin{itemize}
  \item Property tests for everything.
  \item But especially C code.
  \item Good generator coverage is essential.
  \item Timing tests.
  \item Consider supplementing with statistical tests where
    appropriate.
\end{itemize}

\end{frame}

\begin{frame}[fragile]

\begin{minted}[mathescape]{haskell}
prop_verify_timing =
  forAll (arbitrary :: Password) $ \pass ->
    (t, r) <- run . withCPUTime $ verify pass fakeCred
    stop $ conjoin [
        r === NotVerified
      , t >= minHashTime
      ]
  where
    withCPUTime a = do
      t1 <- liftIO getCPUTime
      r <- a
      t2 <- liftIO getCPUTime
      pure (t2 - t1, r)
\end{minted}

\end{frame}

\begin{frame}[fragile]
\frametitle{Some sample ``red flags'' in crypto packages}

\begin{itemize}
\item Using a bespoke implementation of an established primitive for
  no good reason.
\item Trivial (or missing) testsuite.
\item Over-enthusiastic use of `unsafePerformIO`.
\item Derived `Eq` instances for authentication codes or signatures.
\item Unsanitary combination of entropy sources (e.g., sequential
  reading and concatenation).
\end{itemize}

\end{frame}

\begin{frame}[fragile]
\frametitle{Suspicious...}

\centering
\begin{cminted}[mathescape]{haskell}

verifyCredential
  :: Password
  -> Salt
  -> Hash
  -> Verification
\end{cminted}

\end{frame}

\begin{frame}[fragile]
\frametitle{RUN}

\centering
\begin{cminted}[mathescape]{haskell}
verifyCredential
  :: ByteString
  -> ByteString
  -> ByteString
  -> Bool
\end{cminted}

\end{frame}


\begin{frame}

\frametitle{Morals?}

\begin{itemize}

  \item Timing is a side-effect.
  \item Laziness is not always your friend.
  \item It's possible to have too many pure functions.
  \item C is not literally Satan...
  \item ...as long as you have QuickCheck.

\end{itemize}

\end{frame}


\begin{frame}
  \frametitle{Thanks!}


  \centering
  \url{https://github.com/olorin/slides}

  \url{<sio@tesser.org>}
\end{frame}


\end{document}
